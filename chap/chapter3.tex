%%%%%%%%%%%%%%%%%%%%%%%%%%%%%%%%  第3章  %%%%%%%%%%%%%%%%%%%%%%%%%%%%%%%%%%%%%%%%
\chapter{表で遊ぶ}\label{ch:3}
表についてが深いようですので,ここでは比較的基本的だと思われるところだけを紹介します.

\section{表を入れる}
表を入れてみましょう.もうめんどくさいので何も書きません.結果は表\ref{tb:simple}です.この表では,表のカラム宣言である\url{\begin{tabular}}において,lcと書いていますから,左のカラムは左寄せに,右のカラムは中央寄せになっています.各行では,\url{&}記号によってカラムが区切られ\url{\\} で1行が終わります.

\begin{table}[tbh]
  \centering
  \caption{シンプルな表}
  \label{tb:simple}
  \begin{tabular}{lc} %\hline
    column 1 & column 2 \\ \hline \hline
           A & 1        \\
           B & 2        \\ \hline
  \end{tabular}
\end{table}

\section{表をカスタマイズする}
\subsection{罫線をいじる}
ちょっと下品に見えるので個人的にはあまりおすすめしませんが,ガッツリ罫線を入れることも可能です.縦の罫線はカラム宣言の指定子をいじればokです.パイプを突っ込むと縦の罫線が描かれます.結果は表\ref{tb:vertical_lines}です.\par
また,横の罫線は\url{\hline}を挿入すると引かれます.こちらもあまり使いすぎるのは良くないでしょう.

\begin{table}[tbh]
  \centering
  \caption{たくさん罫線を入れた表}
  \label{tb:vertical_lines}
  \begin{tabular}{|c|c|} \hline
    column 1 & column 2 \\ \hline \hline
           A & 1        \\ \hline \hline \hline \hline
           B & 2        \\ \hline
  \end{tabular}
\end{table}


\subsection{列の長さを指定する}
なかなか面倒なやつです.とりあえず第一カラムを長めにしてみましょう.ここでは\url{p{50mm}}としてみました.これは,上/左詰めの幅指定カラムを意味します.結果は表\ref{tb:long_col}となります.


\begin{table}[tbh]
  \centering
  \caption{列の長さを調整する}
  \label{tb:long_col}
  \begin{tabular}{p{50mm}c} %\hline
    column 1                             & column 2 \\ \hline \hline
    長さを指定してみると,勝手に2行になります. & 1        \\
    B                                    & 2        \\ \hline
  \end{tabular}
\end{table}

\subsection{ぶち抜きの表を作る}
\url{\multicolumn}を使えば,複数の列をぶち抜いた表を作成できます.表\ref{tb:multi_col}に結果を示します(効果をわかりやすくするために縦の罫線を引いています).\par

\begin{table}[tbh]
  \centering
  \caption{列をぶち抜く}
  \label{tb:multi_col}
  \begin{tabular}{l|c|c} %\hline
             & \multicolumn{2}{c}{結合部分} \\
    column 1 & column 2 & column 3         \\ \hline \hline
    A        & 1        & 3                \\
    B        & 2        & 4                \\ \hline
  \end{tabular}
\end{table}

では,行をぶち抜く場合はどうでしょう.この場合はプリアンプルで\url{\usepackage{multirow}}しておかなければなりません.文法はmulticolumnとほぼ同じですが,位置指定子はすでにあるものを使うので*としておきます.結果は表\ref{tb:multi_row}のようになります.

\begin{table}[tbh]
  \centering
  \caption{行をぶち抜く}
  \label{tb:multi_row}
  \begin{tabular}{lc} %\hline
    column 1 & column 2                 \\ \hline \hline
           A & \multirow{2}{*}{ぶち抜き} \\
           B &                          \\ \hline
  \end{tabular}
\end{table}