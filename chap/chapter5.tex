%%%%%%%%%%%%%%%%%%%%%%%%%%%%%%%%  第5章  %%%%%%%%%%%%%%%%%%%%%%%%%%%%%%%%%%%%%%%%
\chapter{時系列予測}\label{ch:5}

\section{問題設定}

本論文では,スポーツ興行におけるチケット販売データを対象に,販売構成のモデリングと初動販売情報の活用可能性について検討した.
チケットの価格帯と来場頻度に基づくファン層を組み合わせた25カテゴリ別販売率を用いて,試合ごとの販売傾向を低次元ベクトルとして表現する手法 Game2Vec を提案した.
また,Game2Vec を初動販売データと最終的な販売データの双方に適用し,試合間距離の相関に基づいたベクトル空間の比較により,初動段階の販売傾向が最終的な販売構成を反映していることを確認した.
% さらに,販売初期の試合ベクトルを説明変数として来場予測モデルに組み込み,外的要因のみを用いたモデルと比較することで,leave\mbox{-}one\mbox{-}venue\mbox{-}out 設定において最終的な販売構成および総販売率の予測誤差を低減できることを確認した.
さらに,販売初期の試合ベクトルを説明変数として来場予測モデルに組み込み,外的要因のみを用いたモデルと比較し,最終的な販売構成および総販売率の予測精度の向上を確認した.
これらの結果から,販売構成を潜在ベクトルとして表現するアプローチの,販売傾向の分析と来場予測の両面での有効性を示した.

一方で,提案した Game2Vec は,ある時点までの累積販売構成ベクトルを入力として試合ベクトルを学習しているため,
同じ販売構成に収束した試合同士は,販売プロセスが異なっていても潜在空間上では近接して配置される.
すなわち,どのカテゴリがどの程度売れたかという「最終的な販売構成」は扱えている一方で,
その構成に至るまでの売れ行きのタイミングや増え方の違いまでは明示的には扱えていない.
今後は,累積販売構成だけでなく,日別の増分や販売曲線の形状といった時系列情報もあわせて埋め込みの対象とし,
% 販売プロセスの違いも潜在ベクトルに反映できるモデル拡張を検討したい.
販売プロセスの違いも考慮したモデリング手法の開発に取り組む.
このような時系列的な情報を含めたモデリングにより,どのタイミングで招待・優待を配布すべきか,
どの販売フェーズで価格調整を行うべきかといった実務上の意思決定に直接結び付く指標を提供できる可能性がある.
販売プロセスそのものを精緻に捉えるアプローチは,より現場に近い形での販売最適化や施策設計につながると考えられる.

\section{時系列予測モデル}

% bibtexを使いましょう.文献リストを育てられます.

% \section{bibファイルの準備}
% 拡張子が\url{.bib}のファイルを用意します.名前に合わせて\url{\bibliography{}}コマンドを突っ込みます.例えばファイルが\url{ref.bib}なら\url{\bibliography{ref}}になります.\par
% 中身はbibtexフォーマットの参考文献群になっている必要があります.リスト\ref{list:bib}のような感じのものがたくさん書かれています.
% Google ScholarやIEEE Xplore, ACM Digital Library等はすべてbibtexフォーマットでの参考文献出力に対応しているので,便利に使いましょう.ただし,油断は禁物で,不必要な情報が載っていたり,大文字小文字がめちゃくちゃだったりします.


% \begin{lstlisting}[caption=bibファイル内に記載される項目の例,label=list:bib, language=BibTeX]
% @article{watson1953molecular,
%   title={Molecular structure of nucleic acids},
%   author={Watson, James D and Crick, Francis HC and others},
%   journal={Nature},
%   volume={171},
%   number={4356},
%   pages={737--738},
%   year={1953}
% }
% \end{lstlisting}

% \section{citeによる参照}
% 参照したいところで\url{\cite{<識別名>}}によって参照を加えます.先程の記事を参照\cite{watson1953molecular}してみます.ちなみに参照先がウェブサイトの場合は,参照した時期を記載するとよいでしょう.ここで三毛別羆事件\cite{sankebetsu}を参照してみます.

% \section{その他}
% \url{cite.sty}が利用可能な場合は,複数の参照を一点で行ったときにいい感じにしてくれます.例えば\url{\cite{a,b,c}}としたときに表示が\url{[1][2][3]}ではなく\url{[1-3]}のようになります.