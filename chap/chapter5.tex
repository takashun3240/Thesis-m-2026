%%%%%%%%%%%%%%%%%%%%%%%%%%%%%%%%  第5章  %%%%%%%%%%%%%%%%%%%%%%%%%%%%%%%%%%%%%%%%
\chapter{bibtexのtips}\label{ch:5}
bibtexを使いましょう.文献リストを育てられます.

\section{bibファイルの準備}
拡張子が\url{.bib}のファイルを用意します.名前に合わせて\url{\bibliography{}}コマンドを突っ込みます.例えばファイルが\url{ref.bib}なら\url{\bibliography{ref}}になります.\par
中身はbibtexフォーマットの参考文献群になっている必要があります.リスト\ref{list:bib}のような感じのものがたくさん書かれています.
Google ScholarやIEEE Xplore, ACM Digital Library等はすべてbibtexフォーマットでの参考文献出力に対応しているので,便利に使いましょう.ただし,油断は禁物で,不必要な情報が載っていたり,大文字小文字がめちゃくちゃだったりします.


\begin{lstlisting}[caption=bibファイル内に記載される項目の例,label=list:bib, language=BibTeX]
@article{watson1953molecular,
  title={Molecular structure of nucleic acids},
  author={Watson, James D and Crick, Francis HC and others},
  journal={Nature},
  volume={171},
  number={4356},
  pages={737--738},
  year={1953}
}
\end{lstlisting}

\section{citeによる参照}
参照したいところで\url{\cite{<識別名>}}によって参照を加えます.先程の記事を参照\cite{watson1953molecular}してみます.ちなみに参照先がウェブサイトの場合は,参照した時期を記載するとよいでしょう.ここで三毛別羆事件\cite{sankebetsu}を参照してみます.

\section{その他}
\url{cite.sty}が利用可能な場合は,複数の参照を一点で行ったときにいい感じにしてくれます.例えば\url{\cite{a,b,c}}としたときに表示が\url{[1][2][3]}ではなく\url{[1-3]}のようになります.