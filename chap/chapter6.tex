%%%%%%%%%%%%%%%%%%%%%%%%%%%%%%%%  第6章  %%%%%%%%%%%%%%%%%%%%%%%%%%%%%%%%%%%%%%%%
\chapter{まとめと今後の課題}\label{ch:6}

\section{まとめ}

本論文では,スポーツ興行におけるチケット販売データを対象に,販売構成のモデリングと初動販売情報の活用可能性について検討した.
チケットの価格帯と来場頻度に基づくファン層を組み合わせた25カテゴリ別販売率を用いて,試合ごとの販売傾向を低次元ベクトルとして表現する手法Game2Vecを提案した.
また,Game2Vecを初動販売データと最終的な販売データの双方に適用し,試合間距離の相関に基づいたベクトル空間の比較により,初動段階の販売傾向が最終的な販売構成を一定程度反映していることを確認した.
さらに,販売初期の試合ベクトルを説明変数として来場予測モデルに組み込み,外的要因のみを用いたモデルと比較により,最終的な販売構成および総販売率の予測精度が改善することを示した.
これらの結果から,販売構成を潜在ベクトルとして表現するアプローチが,販売傾向の分析と来場予測の両面で有効である可能性を示唆した.

% 一方で,本論文で得られた知見は,「初動情報が有用である」ことを示すに留まっており,初動販売のどの要素が最終的な販売傾向の予測や説明に寄与しているかについては十分に明らかにできていない.
一方で,本論文では,初動販売情報の有用性を「予測精度の改善」および「埋め込み空間の構造比較」という観点から示したが,どの要素が最終的な販売傾向の差異に寄与しているかといったメカニズムの解明は今後の課題として残る.
たとえば,初動段階においてどのチケット種別・どのファン層のカテゴリが最終的な販売構成の差異を最も強く特徴付けているのか,あるいは初動のどのカテゴリの変化が最終段階のどのカテゴリの増減と結び付いているのかといった検証は行えていない.
また,本論文では販売開始後5日までの初動情報を主として用いたが,初動をどこまで観測すれば最終的な販売傾向をどの程度見通せるのか(情報量と予測可能性の関係)についても十分な検討は行えていない.
これらは,初動情報を用いた意思決定支援を実務に接続する上で重要な論点であり,今後の課題として位置付けられる.


\section{今後の課題}

第一に,初動販売情報の寄与をカテゴリ単位で分析し,初動と最終の対応関係をより詳細に明らかにする必要がある.
具体的には,初動の各カテゴリ(チケット種別$\times$ファン層)における販売率が,最終販売構成のどのカテゴリをどの程度説明するかを分析し,
初動の「どの層・どの価格帯」が最終の傾向差に影響するのかを定量的に示す枠組みが求められる.
この検証により,試合ベクトルが表している販売傾向の解釈性が向上し,施策立案における示唆をより具体化できると考えられる.

第二に,初動情報の観測期間と予測性能の関係を系統的に評価する必要がある.
本論文では初動5日までの情報を用いたが,販売開始後$t$日までの情報を用いた場合($t=1,\ldots,t_{\mathrm{final}}$)に,
最終販売構成や総販売率の予測精度がどのように変化するかを比較によって,
「どの程度の初動情報が得られれば十分な見通しが立つか」という意思決定上の指針を与えられる可能性がある.
さらに,会場や試合条件によって必要な観測期間が異なる可能性もあるため,条件別の分析も重要である.

第三に,本論文のGame2Vecは累積販売構成を入力として試合ベクトルを学習しているため,
同様の販売構成に収束した試合同士は販売プロセスが異なっていても潜在空間上で近接して配置されうる.
すなわち,どのカテゴリがどの程度売れたかという「最終的な販売構成」は捉えられる一方で,
その構成に至るまでの売れ行きのタイミングや増え方の違いを明示的に扱うことは難しい.
今後は,累積販売構成に加えて日別の増分や販売曲線の形状などの時系列情報も埋め込みの対象とし,
販売プロセスの差異を考慮したモデリング手法の開発に取り組む.
このような時系列的情報を含めたモデリングにより,招待・優待を投入すべきタイミングや,
価格調整を行うべき販売フェーズといった実務上の意思決定に直接結び付く指標を提供できる可能性がある.

以上の課題に取り組むことで,初動販売情報を用いた販売最適化の実現に向けて,
より解釈可能で,実務の意思決定に接続しやすいモデリング手法の開発を目指す.









% \chapter{まとめと今後の課題}\label{ch:6}

% \section{まとめ}

% 本論文では,スポーツ興行におけるチケット販売データを対象に,販売構成のモデリングと初動販売情報の活用可能性について検討した.
% チケットの価格帯と来場頻度に基づくファン層を組み合わせた25カテゴリ別販売率を用いて,試合ごとの販売傾向を低次元ベクトルとして表現する手法 Game2Vec を提案した.
% また,Game2Vec を初動販売データと最終的な販売データの双方に適用し,試合間距離の相関に基づいたベクトル空間の比較により,初動段階の販売傾向が最終的な販売構成を反映していることを確認した.
% さらに,販売初期の試合ベクトルを説明変数として来場予測モデルに組み込み,外的要因のみを用いたモデルと比較し,最終的な販売構成および総販売率の予測精度の向上を確認した.
% これらの結果から,販売構成を潜在ベクトルとして表現するアプローチの,販売傾向の分析と来場予測の両面での有効性を示した.

% \section{今後の展望}

% 一方で,提案した Game2Vec は,ある時点までの累積販売構成ベクトルを入力として試合ベクトルを学習しているため,
% 同じ販売構成に収束した試合同士は,販売プロセスが異なっていても潜在空間上では近接して配置される.
% すなわち,どのカテゴリがどの程度売れたかという「最終的な販売構成」は扱えている一方で,
% その構成に至るまでの売れ行きのタイミングや増え方の違いまでは明示的には扱えていない.
% 今後は,累積販売構成だけでなく,日別の増分や販売曲線の形状といった時系列情報もあわせて埋め込みの対象とし,
% 販売プロセスの違いも考慮したモデリング手法の開発に取り組む.
% このような時系列的な情報を含めたモデリングにより,どのタイミングで招待・優待を配布すべきか,
% どの販売フェーズで価格調整を行うべきかといった実務上の意思決定に直接結び付く指標を提供できる可能性がある.
% 販売プロセスそのものを精緻に捉えるアプローチを開発し,より現場に近い形での販売最適化や施策設計を目指す.