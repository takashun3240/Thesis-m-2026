%%%%%%%%%%%%%%%%%%%%%%%%%%%%%%%%  第1章  %%%%%%%%%%%%%%%%%%%%%%%%%%%%%%%%%%%%%%%%
\chapter[はじめに: ここは目次に表示される文章]{はじめに: ここは本文中に表示される文章}\label{ch:1}
thesis.texがすべての取りまとめ役で,chapterX.texが各章に対応します.以下,適当にtipsを交えつつ,適当な文章が続きます.\par

\begin{enumerate}
    \renewcommand{\labelenumi}{(\alph{enumi})}
    \item \texcommand{\labelenumi}をいじるとenumerateの数字を変更できます
    \item \texcommand{\alph}だとa, b, cに,\texcommand{\Roman}だとI, II, IIIになります
    \item よかったらコントロールしてみてね!
\end{enumerate}\par

(a)アルファベットに変更してみました.(b)主に\texcommand{\arabic}(算用数字)や\texcommand{\alph}あたりを使うことになるでしょう.他には\texcommand{\roman}(小文字ローマ数字),\texcommand{\Alph}(大文字アルファベット)もあります.(c)特に書くことがありませんでした.\par
本文中でちょっとだけスペースを開けたいときは,\texcommand{\newline\par}とか書いてみると\newline\par
%\section{研究目的}
%\section{本論文の構成}
こんな感じにスペースが開くかもしれませんね.以降,\ref{ch:2}章では図について少し,\ref{ch:3}章では表について少し書いてみます.\ref{ch:4}章はちょっとしたtipsを書きます.\ref{ch:5}章では,bibtexに触れて,まとめと今後の展望を述べます.
