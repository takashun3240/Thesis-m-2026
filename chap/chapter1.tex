%%%%%%%%%%%%%%%%%%%%%%%%%%%%%%%%  第1章  %%%%%%%%%%%%%%%%%%%%%%%%%%%%%%%%%%%%%%%%
% \chapter[はじめに: ここは目次に表示される文章]{はじめに: ここは本文中に表示される文章}\label{ch:1}
\chapter[はじめに]{はじめに}\label{ch:1}
% thesis.texがすべての取りまとめ役で,chapterX.texが各章に対応します.以下,適当にtipsを交えつつ,適当な文章が続きます.\par

% \begin{enumerate}
%     \renewcommand{\labelenumi}{(\alph{enumi})}
%     \item \texcommand{\labelenumi}をいじるとenumerateの数字を変更できます
%     \item \texcommand{\alph}だとa, b, cに,\texcommand{\Roman}だとI, II, IIIになります
%     \item よかったらコントロールしてみてね!
% \end{enumerate}\par

% (a)アルファベットに変更してみました.(b)主に\texcommand{\arabic}(算用数字)や\texcommand{\alph}あたりを使うことになるでしょう.他には\texcommand{\roman}(小文字ローマ数字),\texcommand{\Alph}(大文字アルファベット)もあります.(c)特に書くことがありませんでした.\par
% 本文中でちょっとだけスペースを開けたいときは,\texcommand{\newline\par}とか書いてみると\newline\par
% %\section{研究目的}
% %\section{本論文の構成}
% こんな感じにスペースが開くかもしれませんね.以降,\ref{ch:2}章では図について少し,\ref{ch:3}章では表について少し書いてみます.\ref{ch:4}章はちょっとしたtipsを書きます.\ref{ch:5}章では,bibtexに触れて,まとめと今後の展望を述べます.

\section{研究背景}

近年,ライブ,フェス,演劇,スポーツといった商業イベントにおいてオンラインでのチケット販売が一般化し,購買日時や購入した席種・価格帯など,チケット販売に関する詳細な時系列データが高い粒度で記録されるようになっている.
このようなデータの蓄積により,イベント運営者が販売過程を定量的に把握し,データに基づいて施策を検討する環境が整いつつある.

商業イベントの運営においては,イベント当日までに発生する需要を見越して,座席(在庫)の配分,販促施策,価格や割引の設計などを調整する必要がある.
特に販売期間中は,販売状況に応じて追加施策を打つなど,逐次的に意思決定を更新することが求められる.
そのため,販売開始後の早い段階で,その後の販売推移や最終的な販売結果を見通す枠組みの重要性が高まっている.

さらに,イベントや会場の新規性(例:新たな会場での開催,新規シリーズの立ち上げ,類似イベントの開催実績が少ない状況など)により,販売開始前に利用できる情報だけでは需要の見積もりが難しいケースもある.
このような状況では,販売開始後に観測される初動販売を手掛かりに,見通しを更新しながら意思決定できる分析・予測の枠組みが有用となる.

本修士論文では,以上の商業イベント一般に共通する意思決定課題を踏まえつつ,第一段階として,購買データが整備され,価格帯や顧客層の違いが運用に直接反映されやすいスポーツ興行を対象に検討を行う.
スポーツ興行は試合ごとに条件が変化し需要が変動しやすいことに加え,ファンの来場頻度に基づく顧客層の違いや,割引・優待・招待などの施策が運用される場合がある.
これらの特徴は,販売構成に着目したモデリングの意義を具体的に議論しやすく,得られた知見を商業イベント一般のチケット販売最適化へ展開するための基盤となる.

\section{チケット販売の現状}

\begin{figure*}[!t]
  \centering
  \begin{minipage}[b]{0.47\linewidth}
    \centering
    % \includegraphics[width=1.0\linewidth]{imgs/game2vec_vis_4days_opponent_rank.pdf}
    \includegraphics[width=1.0\linewidth]{images/ticket_ex1.pdf}
    \subcaption{試合Xの最終販売構成}
    \label{試合X}
  \end{minipage}
  \begin{minipage}[b]{0.47\linewidth}
    \centering
    % \includegraphics[width=1.0\linewidth]{imgs/game2vec_vis_4days_opponent_region.pdf}
    \includegraphics[width=1.0\linewidth]{images/ticket_ex2.pdf}
    \subcaption{試合Yの最終販売構成}
    \label{試合Y}
  \end{minipage}
  \caption{チケット販売構成の比較}
  % \ecaption{Early-sales embedding and its relationship to external factors (opponent ranking and region)}
  \label{販売構成の比較例}
\end{figure*}

\subsection{チケット販売の目標}

商業イベントにおけるチケット販売は,単に販売枚数を最大化するだけでなく,売上や利益といった収益指標に加え,満席に近い会場を実現する販売率,客単価,販売速度,顧客層の構成,来場体験や満足度など,複数の観点を同時に考慮する必要がある.
例えば,高単価席の販売は収益に直結する一方,空席が目立つことは会場の一体感や体験価値に影響しうる.
また,会員向けの先行販売や特典設計は短期売上だけでなく中長期の顧客関係にも関わる.
このように,チケット販売は複数KPIのバランスを取りながら運用される意思決定課題として位置づけられる.

運営が調整可能な施策としては,券種・席種の設計,価格設定,割引やキャンペーンの投入,販売チャネルや露出の最適化,団体販売の強化などが挙げられる.
さらに販売期間中の状況に応じて,特定の価格帯・席種の訴求を強める,残席の多いエリアに施策を集中する,追加プロモーションを行うといった調整も行われる.
施策は総販売数に影響するだけでなく,どの価格帯が売れるか,どの顧客層が購入するか,購入のタイミングがいつになるかといった販売構造そのものにも影響する.

\subsection{意思決定の流れ}

チケット販売の意思決定は大きく「販売開始前の計画」と「販売期間中の運用」に分けられる.
販売開始前には,想定需要に基づき,価格帯,席種配分,販売スケジュール(先行販売・一般販売等),プロモーション計画を設計する.
一方,販売期間中には日々の販売状況を観測し,計画からの乖離に応じて追加施策を検討する.
とりわけ販売開始直後から一定期間の初動は,イベント固有の需要を反映した重要な情報となり,その後の施策判断(追加露出,割引の投入,席種訴求の変更など)を更新する根拠として用いられる.

また,新規会場での開催や新規シリーズの立ち上げ,出演者やコンテンツの組み合わせが過去と大きく異なるイベントなど,過去データに基づく事前推定が難しい条件では,販売開始前の情報だけで需要を見通すことは困難になりやすい.
このような状況では,販売開始後に観測される初動販売データを手掛かりとして見通しを更新し,意思決定に結び付ける分析枠組みの重要性が高まる.

\subsection{データ活用の課題}

販売状況の把握には,累積販売数や日別販売数といった基本指標に加え,価格帯・席種別の販売速度,残席の分布,顧客層別の購入比率,販売チャネル別の構成,複数枚購入の割合などが参照される.
また,販売開始前に得られる情報(イベント内容,開催日時,季節性,告知規模,過去の類似イベントの販売実績など)も判断材料となる.
これらのデータは,販売が計画どおりに進んでいるか,どの価格帯・どの層に偏りがあるか,残席リスクがどこにあるかといった観点から意思決定を支える.

しかし,販売の現場では多様なデータが蓄積されている一方で,それらを統合して意思決定に直接結び付ける形で活用することは容易ではない.
個々の指標は参照されるものの,イベント間で比較可能な共通表現として整理されず,初動段階の情報が最終的な販売結果にどの程度結び付くかを定量的に評価する枠組みも十分ではない.
その結果,施策判断は経験則に依存しやすく,判断根拠の説明や再利用が難しくなる.
このような背景から,販売過程や販売構成を比較可能な形で表現し,初動から最終結果を見通すためのデータモデリング手法が求められる.


\section{課題}

前節では,チケット販売において,販売期間中の状況を観測しながら施策判断が更新されること,また販売開始後の初動販売が見通し更新の重要な手掛かりになりうることを述べた.
一方で,初動を意思決定に活用するためには,初動段階で得られる情報から「どのような販売傾向のイベントなのか」を把握し,最終的な販売結果との関係を適切に捉える必要がある.
しかし,現状ではそのための枠組みが十分に整備されていない.

まず,チケット販売の状況を把握する際に,最終的な販売枚数(来場者数)や販売率といった総量指標だけでは,価格帯や顧客層といった内訳情報が失われるため,施策と対応付けた解釈が難しい.
特に販売期間中に調整可能な施策(例:割引やプロモーション,特定席種の訴求)は,総量だけでなく「どの価格帯がどの層に売れているか」といった販売の偏りにも影響しうる.
したがって,初動を含む販売状況を理解する上では,総量ではなく販売構成(価格帯・顧客層などの内訳)を保持した形で捉える必要がある.

一方で,販売構成(価格帯や顧客層ごとの内訳)をそのまま用いてイベント(試合)間の類似性を議論しようとすると,どのカテゴリの差異を重視すべきかが明確でなく,結果の比較・解釈が一貫しにくい場合がある.
特に販売開始直後の初動段階では,カテゴリによって販売枚数が0となることも多く,単純な内訳ベクトルの比較は疎性の影響を受けやすい.
さらに,会場規模の違いにより販売数の絶対値を直接比較しにくい場合もある.
このため,販売構成の情報を保持しつつ,初動の疎なデータも含めてイベント間で比較可能に扱うための整理方法が求められる.

次に,初動段階の販売構成が,最終的な販売傾向をどの程度反映しているかは自明ではない.
初動の段階で似た売れ方を示すイベントが,最終的にも似た販売傾向に収束するとは限らないため,「販売開始後何日程度の情報が得られれば最終傾向を見通せるのか」を定量的に判断することは難しい.
したがって,初動と最終の関係を,イベント間で比較可能な尺度に基づいて評価し,初動がどの程度最終傾向を捉えているかを明らかにする必要がある.

さらに,来場予測・需要予測に関する既存研究の多くは最終的な来場者数(総販売数)を予測対象としており,販売開始後に得られる初動情報,とくに販売構成の情報を,扱いやすい形で予測に取り込む枠組みは十分に整理されていない.
過去実績が乏しい条件では販売開始前の情報だけでは不確実性が大きくなりうるため,初動情報を予測に活用できる形で特徴量化し,外的要因のみの場合と比較して有効性を検証することが重要となる.

以上の課題から,本研究では販売開始後の初動段階に得られる販売構成に着目し,イベント間で比較可能な形へ整理した上で,最終的な販売結果との関係および来場予測への応用可能性を検討する.

\section{本論文の貢献と構成}


\begin{figure*}[t]
    \centering
    \includegraphics[width=\textwidth]{images/model_overall_ver15.pdf}
    \caption{販売データモデリングの概要}
    % \ecaption{The overview for purchase data modeling.}
    % \Description{An illustration of a player tracking pipeline with multiple stages including detection, tracking, and ID correction.}
    \label{fig:overview}
\end{figure*}

% % 前節で整理した課題意識を踏まえ,本研究では販売開始後の初動段階に得られる販売構成に着目し,イベント(試合)ごとの販売傾向を比較可能な形で扱う枠組みを検討する.

% そこで本研究では,試合(イベント)ごとの販売構成をベクトル表現として扱うための手法 Game2Vec を提案する.
% Game2Vec は,チケットの価格帯および顧客層(来場頻度に基づく区分)に基づいて販売構成を定義し,イベントごとの販売傾向を低次元の潜在表現(試合ベクトル)へ整理する枠組みである.
% これにより,販売構成に含まれる内訳情報を保持しつつ,イベント間で比較可能な形で販売傾向を扱うことを目指す.
% また,会場規模の違いなどにより販売数の絶対値を直接比較しにくい状況を踏まえ,比較可能性を担保するための前処理を行った上で表現を構築する.

% 本研究では,販売開始後の任意の時点までのデータから得られる試合ベクトルと,最終販売結果に基づく試合ベクトルの関係を検討する.
% 具体的には,販売開始後t日までのデータに基づいて構成したベクトル空間と,最終販売に基づいて構成したベクトル空間を比較し,初動段階の情報が最終的な販売傾向をどの程度捉えているかを定量的に評価する.
% この比較により,「販売開始後どの程度の期間のデータが得られれば,最終的な販売傾向を見通す手掛かりになり得るか」を検討する.

% さらに,得られた試合ベクトルを来場予測における説明変数として利用し,販売開始前に得られる外的要因のみを用いる場合と比較することで,販売開始後の初動情報を取り込むことの有効性を検証する.
% とくに,過去実績が乏しい条件を含む状況では販売開始前情報に基づく推定の不確実性が大きくなりうるため,初動情報が予測精度の改善に寄与するかを検討する.

本研究では,試合ごとの販売構成をベクトル表現として扱うための手法Game2Vecを提案する.
本手法では,チケット種別(有料・優待・招待)の価格帯と,購入者の来場頻度に基づいて定義したファン層を組み合わせ,販売データを25カテゴリの構成比として扱う.
このようなカテゴリ別販売数をそのまま扱うだけでは,試合間の関係性や典型的な販売パターンを十分に把握することが難しい.
そこでGame2Vecでは,これらの販売構成に内在する共通パターンや特徴的な差異を学習し,試合ごとの販売傾向を圧縮した潜在表現へと変換する.
この潜在ベクトルにより,似た販売傾向を持つ試合同士が空間上で近接し,異なる傾向をもつ試合は離れて配置されるようになる.
Game2Vec により各試合の販売構成から得られる潜在表現は,試合ごとの販売傾向を要約した表現であるため,以降,本研究ではこれを「試合ベクトル」と呼ぶ.
このアプローチは,単語やエリア,プレーヤなど多様な対象をベクトル空間に埋め込み,下流タスクに活用してきた分散表現学習の流れに位置付けられる\cite{church2017word2vec}\cite{shoji2024area}.

図\ref{fig:overview}に,本研究による販売データモデリングの概要を示す.
図に示すように,販売開始後の任意の日数までのデータを用いて試合ごとの販売構成を作成し,これをベクトルとして埋め込むことで,任意の期間までの販売傾向を埋め込んだベクトル空間を作成した.
これらのベクトル空間と最終的な販売結果で埋め込んだベクトル空間の比較により,最終的な販売傾向が販売開始後どの時点から出ていたかを検証した.
具体的には,両者のベクトル空間における試合ペア間の距離を利用した距離構造の一致度を利用して,ある期間までの販売データが最終的な販売傾向をどの程度捉えているかを評価した.
その結果,販売開始直後の初動データから得られる試合ベクトルが,最終的な販売傾向を一定捉えていることを明らかにした.

また,Game2Vec によって得られた試合ベクトルの予測タスクへの有効性を検証するために,試合ベクトルの各次元を来場予測モデルの説明変数として組み込み,外的要因のみに基づく従来モデルとの比較実験を行った.
その結果,試合ベクトルを追加したモデルは,外的要因のみのモデルと比べて最終的な販売構成および総販売率(25カテゴリ別販売率の合計)の予測精度が改善されることを確認した.
今回の来場予測実験では,会場単位で学習と検証を分離して,予測対象の会場の販売実績が学習データに含まれないようにした.
この検証設定は,過去データが乏しく外的要因に基づく予測の信頼性が低い条件を想定している.
このような状況こそ,初動の販売傾向を埋め込んだ試合ベクトルの説明変数としての活用が重要となる場面であり,本研究の実験結果は,そうした条件下で試合ベクトルの予測精度改善への有用性を示している.
この枠組みの利用により,実際の運営現場で初動販売の状況を観測しながら,招待券の配布や販促の追加実施など,動的な販売最適化の支援が可能となる.


本論文の構成を以下に示す.
第2章では関連研究を整理し,本研究の位置づけを明確にする.
第3章ではデータセットおよび販売構成(カテゴリ設計)と前処理を述べる.
第4章では提案手法 Game2Vec のモデル構造と学習方法を説明する.
第5章では初動・最終の比較実験および来場予測への応用実験を行い,結果を考察する.
第6章で結論と今後の課題を述べる.

% \section{本論文の構成}


% %%%%% 以下がver3の序論 %%%%
% \section{研究背景と課題}

% % スポーツ興行のオンライン販売化が進み,チケット販売に関する詳細な時系列データが蓄積されるようになった.
% スポーツの試合観戦におけるチケットのオンライン販売化が進み,チケット販売に関する詳細な時系列データが蓄積されるようになった.
% また,デジタルプラットフォームの導入により,購入者の属性や購入したチケットの価格帯,日別の販売数などが高い粒度で記録され,販売過程を定量的に分析する環境が整いつつある.
% このような環境の変化は,スポーツチームが販売データを基にした戦略的なチケット販売の最適化を行うための基盤となっている\cite{Dominik2022}.

% さらに,新スタジアムの開設や大規模な改修が相次いでおり,過去データが十分に蓄積されていない会場において来場予測が求められるケースが増えている\cite{EPSI2024Stadiums}\cite{JSA2025StadiumsRep}.
% このような環境では,従来のような外的要因(曜日,対戦相手,天候,チーム成績など)に基づいて事前に来場者数を推定する手法では精度が低下しやすい\cite{hecksteden2022smallsamples}.
% % そのため,販売開始直後の初動販売の状況を観察しながら,その後の販売推移や最終的な販売構成を早期に把握することの重要性が高まっている.
% そのため,販売開始直後の初動販売の状況を観察しながら,その後の販売推移や最終的な販売構成を早期に把握することの重要性が高まっている.
% しかし,初動販売が最終的な販売傾向とどの程度関連しているのかを定量的に評価する枠組みは,これまで十分には整備されてこなかった.

% 初動段階から最終的な販売傾向を捉えようとするためには,単に「最終的に何枚売れたか」を扱うだけでは不十分である.
% そこでまず,販売開始から試合当日までに売れ行きがどのようなペースで変化したのかという販売プロセスに注目する必要がある.
% % しかし,総販売枚数の推移だけでは試合ごとの「売れ方」の違いや,初動と最終結果の関係を適切に比較することが難しい.
% しかし,総販売枚数の推移だけでは試合ごとの「売れ方」の違いや,初動と最終結果の関係の比較が難しい.
% % そのため,どの価格帯・チケット種別・ファン層のチケットがどの程度売れたのかといった販売構成の利用が不可欠である.
% % そのため,どの価格帯のチケットがどのファン層にどれほど売れたのかといった販売構成の利用が不可欠である.
% そのため,どんなチケットが誰にどれほど売れたのかといった販売構成の利用が不可欠である.

% そこで本研究では,試合ごとの販売構成をベクトル表現として扱うための手法 Game2Vec を提案する.
% 本手法では,チケット種別(有料・優待・招待)の価格帯と,購入者の来場頻度に基づいて定義したファン層を組み合わせ,販売データを25カテゴリの構成比として扱う.
% % そこで本研究では,チケット販売データを「最終的な売上枚数」だけでなく,チケット種別(有料/優待/招待)の価格帯と来場頻度に基づくファン層を組み合わせた25カテゴリ別販売率として試合ごとの販売構成を表現し,その販売傾向の違いをベクトル空間に埋め込む手法 Game2Vec を提案する.
% このようなカテゴリ別販売数をそのまま扱うだけでは,試合間の関係性や典型的な販売パターンを十分に把握することが難しい.
% % そこで,Game2Vecにより各試合の販売構成を低次元の潜在ベクトルとして表現することで,販売傾向の違いを可視化やクラスタリングによって解釈しやすくするとともに,下流の来場予測モデルにおける説明変数としても利用可能になる.
% % そこで,Game2Vecにより各試合の販売構成を低次元の潜在ベクトルとして表現することで,似た販売構成を持つ試合はベクトル空間上で近くに配置される.
% % そこで,Game2Vecにより各試合の販売構成を低次元の潜在ベクトルとして表現する.
% そこでGame2Vecでは,これらの販売構成に内在する共通パターンや特徴的な差異を学習し,試合ごとの販売傾向を圧縮した潜在表現へと変換する.
% この潜在ベクトルにより,似た販売傾向を持つ試合同士が空間上で近接し,異なる傾向をもつ試合は離れて配置されるようになる.
% % 本研究では,Game2Vec により各試合の販売構成から得られる潜在表現を,以降「試合ベクトル」と呼ぶ.
% Game2Vec により各試合の販売構成から得られる潜在表現は,試合ごとの販売傾向を要約した表現であるため,以降,本研究ではこれを「試合ベクトル」と呼ぶ.
% % ここでは,それぞれの試合での販売構成が持つ内在パターンを学習することで,似た販売構成を持つ試合はベクトル空間上で近くに配置される.
% % このアプローチは,単語やエリア,プレーヤなど多様な対象をベクトル空間に埋め込み,下流タスクに活用してきた分散表現学習の流れに位置付けられるものであり\cite{church2017word2vec}\cite{shoji2024area},チケット販売の構造情報にも同様の枠組みを適用するものである.
% このアプローチは,単語やエリア,プレーヤなど多様な対象をベクトル空間に埋め込み,下流タスクに活用してきた分散表現学習の流れに位置付けられる\cite{church2017word2vec}\cite{shoji2024area}.

% 図\ref{fig:overview}に,本研究による販売データモデリングの概要を示す.
% 図に示すように,販売開始後の任意の日数までのデータを用いて試合ごとの販売構成を作成し,これをベクトルとして埋め込むことで,任意の期間までの販売傾向を埋め込んだベクトル空間を作成した.
% これらのベクトル空間と最終的な販売結果で埋め込んだベクトル空間の比較により,最終的な販売傾向が販売開始後どの時点から出ていたかを検証した.
% 具体的には,両者のベクトル空間における試合ペア間の距離を利用した距離構造の一致度を利用して,ある期間までの販売データが最終的な販売傾向をどの程度捉えているかを評価した.
% その結果,販売開始直後の初動データから得られる試合ベクトルが,最終的な販売傾向を一定捉えていることを明らかにした.

% % さらに,初動販売データによるベクトル空間と最終的な販売データによるベクトル空間の類似性を分析した.
% % 具体的には,両者のベクトル空間における試合ペア間のユークリッド距離を比較し,距離構造の一致度から初動ベクトルが最終的な販売傾向をどの程度捉えているかを評価した.
% % その結果,販売開始直後の初動データから得られる試合ベクトルが,最終的な販売傾向をよく説明していることを示した.
% % また,Game2Vec によって得られた試合ベクトルを説明変数として下流の予測モデルに組み込み,外的要因のみに基づく従来モデルとの比較実験を行った.
% % また,Game2Vec によって得られた試合ベクトルの説明変数としての有効性を検証するために来場予測モデルに組み込み,外的要因のみに基づく従来モデルとの比較実験を行った.
% また,Game2Vec によって得られた試合ベクトルの予測タスクへの有効性を検証するために,試合ベクトルの各次元を来場予測モデルの説明変数として組み込み,外的要因のみに基づく従来モデルとの比較実験を行った.
% % その結果,試合ベクトルを追加したモデルは,外的要因のみのモデルと比べて最終的な販売構成および総販売率(25カテゴリ別販売率の合計)の予測精度が改善されることを確認した.
% その結果,試合ベクトルを追加したモデルは,外的要因のみのモデルと比べて最終的な販売構成および総販売率(25カテゴリ別販売率の合計)の予測精度が改善されることを確認した.
% % その結果,試合ベクトルを追加したモデルは,外的要因のみのモデルと比べて最終的な販売構成および総販売率(25カテゴリ別販売率の合計)の予測誤差が小さく,予測精度が改善されることを確認した.
% 今回の来場予測実験では,会場単位で学習と検証を分離して,予測対象の会場の販売実績が学習データに含まれないようにした.
% この検証設定は,過去データが乏しく外的要因に基づく予測の信頼性が低い条件を想定している.
% このような状況こそ,初動の販売傾向を埋め込んだ試合ベクトルの説明変数としての活用が重要となる場面であり,本研究の実験結果は,そうした条件下で試合ベクトルの予測精度改善への有用性を示している.
% この枠組みの利用により,実際の運営現場で初動販売の状況を観測しながら,招待券の配布や販促の追加実施など,動的な販売最適化の支援が可能となる.

% \section{本論文の構成}

% 本研究による貢献は次の3つである.

% \begin{itemize}
% % \item チケットの価格帯と購入者層に基づく販売構成を用いて,試合ごとの販売傾向をベクトル空間に埋め込む手法「Game2Vec」を提案し,販売構成の比較や可視化を可能にした.
% \item チケットの価格帯とファン層に基づく販売構成を用いて,試合ごとの販売傾向をベクトル空間に埋め込む手法「Game2Vec」を提案した.
% % \item Game2Vecを初動販売と最終的な販売データに適用し,それぞれのベクトル空間を比較することで,各試合の初動段階での販売傾向が,最終的な販売傾向を捉えていることを示した.
% % \item Game2Vecを初動販売と最終的な販売データに適用し,それぞれのベクトル空間を比較することで,各試合の初動段階での販売傾向が,最終的な販売傾向をどの程度捉えているかを定量的に示した.
% \item Game2Vecを初動販売と最終的な販売データに適用し,それぞれのベクトル空間の比較により,各試合の初動段階での販売傾向が,最終的な販売傾向をどの程度捉えているかを定量的に示した.
% % \item 販売初期のベクトル表現を活用した予測と外的要因のみを用いた予測を比較し,Game2Vecによる埋め込み表現が予測タスクでも有効であることを実証した.
% \item 販売初期のベクトル表現を活用した予測と外的要因のみを用いた予測を比較し,Game2Vecによる埋め込み表現の予測タスクでの有効性を確認した.
% \end{itemize}
