%%%%%%%%%%%%%%%%%%%%%%%%%%%%%%%%  第2章  %%%%%%%%%%%%%%%%%%%%%%%%%%%%%%%%%%%%%%%%
\chapter{関連研究}\label{ch:2}

\section{外的要因における来場予測}

スポーツ興行におけるチケット販売予測は,長年にわたり多くの研究が行われてきた.
従来の予測モデルの多くは,試合の曜日,対戦相手,チームの成績,スタジアムの規模,天候といった外的要因を説明変数とし,
試合ごとの最終的な販売枚数や収容率の予測を目的としている\cite{pang2024forecasting}\cite{mueller2020pre}.
たとえば,Pang ら\cite{pang2024forecasting}は,複数の外的要因を入力とした機械学習モデルを構築し,
試合単位の最終的な観客数の予測により,販売戦略立案への活用可能性を示している.
また,Mueller ら\cite{mueller2020pre}は,価格設定やプロモーション情報といった要因を組み込んだ来場予測モデルを提案し,
価格や割引施策が入場者数に与える影響を分析している.

これらの研究は,外的要因に基づいてイベント全体の需要水準を予測するうえで有効であり,
試合の「どれだけ売れるか」を事前に把握するための枠組みを提供している.
一方で,どのようなチケットがどのような来場者層に販売されたのかといった販売の「構造的な内訳」や,
販売プロセスにおける「時間的推移」を扱うものは限られている.


\section{販売過程の時間推移分析}

チケット販売における時間的な側面を考慮した研究もいくつか報告されている.
Pang ら\cite{pang2025time}は,シーズンを通じた試合ごとの最終来場率を時系列データとみなし,LSTM を用いてその推移を予測している.
また,Shapiro ら\cite{shapiro2021examination}は,プレシーズン時や試合2週間前,1週間前,前日といった複数時点におけるチケット価格を外的要因から予測し,価格の時間変化と要因との関係を分析している.
いずれの研究も時間情報をモデルに取り込んではいるものの,本研究が対象とするような,各試合の販売開始から試合当日までの売れ行きの推移や,その過程における販売構成の違いをモデリングする研究とは異なる.

販売過程の時系列的なモデリングは,ホテル予約や航空券販売など他ドメインではより活発に研究されている.
Fiori ら\cite{fiori2019reservation}や Contessi ら\cite{contessi2024decoding}はホテル予約データを対象に予約曲線を分析し,
早期予約情報から最終需要を推定する枠組みを提案している.
また,Imanaka ら\cite{imanaka2021predicting}や Gao ら\cite{gao2022dynamic}は航空券販売の時系列データを用いて,
販売初期の予約状況や価格推移から最終的な搭乗需要を予測するモデルを構築している.
しかし,スポーツイベントのチケット販売において,試合単位での販売数の推移やチケット種別・ファン層を含む販売構成を時系列的にモデル化し,
販売初期と最終結果との関係を試合ごとに比較可能な形で捉える枠組みは,依然として十分に検討されていない.

\section{販売構成(チケット種別・ファン層)の分析}

販売枚数の総量だけでなく,価格帯やファン属性といった販売構成に着目した研究も報告されている\cite{Popp2025price}.
Humphreys ら\cite{humphreys2022separating}は,観客をホームファンとアウェイファンに分類し,
それぞれの来場決定に影響する要因分析により,ファン属性による来場行動の違いを明らかにしている.
また,Quansah ら\cite{quansah2024determining}は,チケット価格の決定要因を分析し,
価格設定に影響を与える外的要因や市場要因を明らかにしている.
さらに,Kaiser ら\cite{kaiser2019well}は,スタジアム内の座席を価格帯ごとのカテゴリに分けたうえで,
カテゴリごとに試合単位の支払意思額(Willingness To Pay)を推定し,価格帯別の需要構造を評価している.
このように,既存研究では,ファン属性や価格帯といった個別の観点から販売構成を分析する試みは行われている.
しかし,これらの多くは特定の時点や集計値として販売構成を捉えており,時間的な変化は明示的には扱っていない.

\section{本研究の位置付け}

従来研究は「外的要因に基づく最終来場数の予測」,「販売過程の時間推移のモデリング」,「価格帯やファン層といった販売構成の分析」といった個別の側面については多くの知見を蓄積しているが,価格帯,来場頻度に基づくファン層といった販売構成を一体的に扱いつつ,
それが販売期間の中でどのように形成・変化していくのかを試合単位でモデリングする手法は,まだ提案されていない.
そこで本研究では,このギャップに対して,価格帯とファン層からなる 25 カテゴリ別販売率に基づいて試合ごとの販売構成を潜在ベクトルとして表現し,
初動販売と最終販売の関係性の分析および来場予測への活用を可能にする Game2Vec を提案する.

% この章では,図について少し書いてみます.thesis.texのプリアンプルでもごちゃごちゃしていますが,ここでは基本的にjpgやpngの図を載せる方針です.eps図やpdf図は誰か詳しい人がいたらこの文章をアップデートしてみてください.

% \section{図をLaTeXで使う}
% thesis.texのプリアンプルを見てみましょう.\url{\usepackage[dvipdfmx]{graphicx}}と書かれていませんか? TeXはそれ自身が画像の取り込みをサポートしているわけではありません.この場合,dvipdfmx (ドライバ)の力を借りています.このドライバにはdvipdfmというものもあり,昔は挙動が違いました(現在,dvipdfmはディスコンで,実体はdvipdfmxとなっているはずです.コマンドラインで\url{$dvipdfm} \url{--version}すると確認できるでしょう).\par
% dvipdfmxでは,jpg・png・pdfではextractbbを実行し勝手にバウンディングボックスを作成してくれます.epsの場合はなんかいい感じにしてくれるでしょう.図はそのまま最終的な出力pdfに突っ込まれることになるので,サイズ等に気をつけたほうがよいかもしれません.


% \section{普通に図を入れる}
% この節では,いろいろ試しつつ図を入れてみます.

% \subsection{まずは一枚}\label{ch:2-single_graphics}
% さて,とりあえずは一枚図を入れてみましょう.ソースは以下のリスト\ref{list:single_graphic}のようになるはずです.表示された図は\ref{fig:a_figure}になっています.このフォーマットではおそらく\SI{15}{\centi\metre}が最大の幅で,それ以上では本文の幅を超過した状態になるでしょう.

% \begin{lstlisting}[caption=図を一枚入れる,label=list:single_graphic, language=TeX]
% \begin{figure}[tbh]
%   \centering
%   \includegraphics[width=8cm]{images/dummy.png}
%   \caption{一枚だけ図を入れてみる}
%   \label{fig:a_figure}
% \end{figure}
% \end{lstlisting}

% \begin{figure}[tbh]
%   \centering
%   \includegraphics[width=8cm]{images/dummy.png}
%   \caption{1枚だけ図を入れてみる}
%   \label{fig:a_figure}
% \end{figure}

% \section{複数枚の図を並べる}
% このファイルではsubcaptionを使って複数の図を処理します.subfigやsubfigureは非推奨のようです.とりあえず複数の図を突っ込んでみましょう.もうソースを載せるのはたるいので,コードを直接読んでみてください.結果は図\ref{fig:fig_set}に示しています.本文中で個々の図を参照すると図\ref{fig:fig_part_1}のように参照されます.また,図のうちサブで振られた文字を取りたい場合は\subref{fig:fig_part_1}とするようです.

% \clearpage

% \begin{figure}[tbh]
%   \centering %これ必要かわからない
%   % 最初の図---------------------------
%   \begin{minipage}[b]{0.45\hsize}
%     \centering
%     \includegraphics[width=5cm]{images/dummy.png}
%     \subcaption{何かしらの図のひとつ目(ここが長すぎると2枚の図がズレるので,遭遇した人は頑張ってググるなりして直してください)}
%     \label{fig:fig_part_1}
%   \end{minipage}
%   % 2番目の図--------------------------
%   \begin{minipage}[b]{0.45\hsize}
%     \centering
%     \includegraphics[ width=5cm]{images/dummy.png}
%     \subcaption{何かしらの図のふたつ目}
%     \label{fig:fig_part_2}
%   \end{minipage}
%   \caption{2枚並べた図}
%   \label{fig:fig_set}
% \end{figure}

% \section{図に,キャプション以外の文字列を追加する}
% プリアンプルでccaptionが読み込んでみました.このパッケージはキャプション以外にちょっと説明文をつけたいなあという場合に役立つでしょう.出力結果は,図\ref{fig:using_ccaption}のようになります.

% \begin{figure}[tbh]
%   \centering
%   \includegraphics[width=8cm]{images/dummy.png}
%   \caption{キャプションは短くしたい}
%   \legend{ここに長い説明が書けるようになります!!!!!!!!!!!!!!!!!!!!}
%   \label{fig:using_ccaption}
% \end{figure}

% \section{図のキャプションに,参考文献を入れつつ,本文中の引用番号を昇順にそろえる}

% caption の [] には図目次・表目次に入れる表記を書いて,{} には本文中に書く cite 入りの表記を書く.

% \begin{figure}[tbh]
%   \centering
%   \includegraphics[width=8cm]{images/dummy.png}
%   \caption[citeを入れない]{citeを入れる→\cite{watson1953molecular}}
%   \label{fig:using_ccaption}
% \end{figure}