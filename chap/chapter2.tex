%%%%%%%%%%%%%%%%%%%%%%%%%%%%%%%%  第2章  %%%%%%%%%%%%%%%%%%%%%%%%%%%%%%%%%%%%%%%%
\chapter{関連研究}\label{ch:2}

\section{外的要因における来場予測}

スポーツ興行におけるチケット販売予測は,長年にわたり多くの研究が行われてきた.
従来の予測モデルの多くは,試合の曜日,対戦相手,チームの成績,スタジアムの規模,天候といった外的要因を説明変数とし,
試合ごとの最終的な販売枚数や収容率の予測を目的としている\cite{pang2024forecasting}\cite{mueller2020pre}.
たとえば,Pang ら\cite{pang2024forecasting}は,複数の外的要因を入力とした機械学習モデルを構築し,
試合単位の最終的な観客数の予測により,販売戦略立案への活用可能性を示している.
また,Mueller ら\cite{mueller2020pre}は,価格設定やプロモーション情報といった要因を組み込んだ来場予測モデルを提案し,
価格や割引施策が入場者数に与える影響を分析している.

これらの研究は,外的要因に基づいてイベント全体の需要水準を予測するうえで有効であり,
試合の「どれだけ売れるか」を事前に把握するための枠組みを提供している.
一方で,どのようなチケットがどのような来場者層に販売されたのかといった販売の「構造的な内訳」や,
販売プロセスにおける「時間的推移」を扱うものは限られている.


\section{販売過程の時間推移分析}

チケット販売における時間的な側面を考慮した研究もいくつか報告されている.
Pang ら\cite{pang2025time}は,シーズンを通じた試合ごとの最終来場率を時系列データとみなし,LSTM を用いてその推移を予測している.
また,Shapiro ら\cite{shapiro2021examination}は,プレシーズン時や試合2週間前,1週間前,前日といった複数時点におけるチケット価格を外的要因から予測し,価格の時間変化と要因との関係を分析している.
いずれの研究も時間情報をモデルに取り込んではいるものの,本研究が対象とするような,各試合の販売開始から試合当日までの売れ行きの推移や,その過程における販売構成の違いをモデリングする研究とは異なる.

販売過程の時系列的なモデリングは,ホテル予約や航空券販売など他ドメインではより活発に研究されている.
Fiori ら\cite{fiori2019reservation}や Contessi ら\cite{contessi2024decoding}はホテル予約データを対象に予約曲線を分析し,
早期予約情報から最終需要を推定する枠組みを提案している.
また,Imanaka ら\cite{imanaka2021predicting}や Gao ら\cite{gao2022dynamic}は航空券販売の時系列データを用いて,
販売初期の予約状況や価格推移から最終的な搭乗需要を予測するモデルを構築している.
しかし,スポーツイベントのチケット販売において,試合単位での販売数の推移やチケット種別・ファン層を含む販売構成を時系列的にモデル化し,
販売初期と最終結果との関係を試合ごとに比較可能な形で捉える枠組みは,依然として十分に検討されていない.

\section{販売構成(チケット種別・ファン層)の分析}

販売枚数の総量だけでなく,価格帯やファン属性といった販売構成に着目した研究も報告されている\cite{Popp2025price}.
Humphreys ら\cite{humphreys2022separating}は,観客をホームファンとアウェイファンに分類し,
それぞれの来場決定に影響する要因分析により,ファン属性による来場行動の違いを明らかにしている.
また,Quansah ら\cite{quansah2024determining}は,チケット価格の決定要因を分析し,
価格設定に影響を与える外的要因や市場要因を明らかにしている.
さらに,Kaiser ら\cite{kaiser2019well}は,スタジアム内の座席を価格帯ごとのカテゴリに分けたうえで,
カテゴリごとに試合単位の支払意思額(Willingness To Pay)を推定し,価格帯別の需要構造を評価している.
このように,既存研究では,ファン属性や価格帯といった個別の観点から販売構成を分析する試みは行われている.
しかし,これらの多くは特定の時点や集計値として販売構成を捉えており,時間的な変化は明示的には扱っていない.

% \subsection{潜在空間への埋め込みモデルの応用}

% 自然言語処理や推薦システムの分野では,分散表現に基づいて対象をベクトル空間に埋め込み,類似性や構造を定量的に扱う手法が広く用いられている.
% Word2Vec\cite{church2017word2vec}に代表される単語埋め込みモデルは,共起関係に基づいて単語を低次元の潜在ベクトルとして表現し,
% 意味的に類似した単語どうしがベクトル空間上で近接することを示している.
% この発想は,都市内の滞在行動を対象とした Area2Vec\cite{shoji2024area} や,
% 生活行動データを対象とした応用研究\cite{shoji2025life}など,都市解析や行動分析の分野にも広がっている.

% スポーツ分野でも,選手やプレーの特徴をベクトル化する試みがなされている.
% Zhang ら\cite{zhang2019player2vec}は,試合中のプレーシーケンスに基づいて選手を潜在空間に埋め込み,
% プレースタイルの類似性を分析する Player2Vec を提案している.
% さらに,購買データを対象に,商品や顧客をベクトル表現として学習する研究\cite{kohei2022}も報告されており,
% 複雑な購買行動のパターンを低次元の潜在空間で捉えるアプローチが広がりつつある.

% これらの研究は,高次元かつ構造的な情報をベクトル空間に埋め込むことで,
% 類似性の可視化やクラスタリング,下流タスクにおける予測性能の向上を実現している.
% しかし,スポーツイベントのチケット販売において,チケット種別やファン層を含む販売構成そのものを潜在ベクトルとして表現し,
% 試合ごとの販売傾向をベクトル空間上で比較可能にする枠組みはほとんど提案されていない.
% 本研究で提案する Game2Vec は,こうした分散表現の枠組みをチケット販売データに適用し,
% 販売構成に基づいて試合を潜在ベクトルとして埋め込む点に特徴がある.

\subsection{潜在空間への埋め込みモデルの応用}

本研究で用いる Game2Vec は,対象となる試合を低次元の潜在ベクトルとして表現する点で,分散表現に基づく埋め込みモデルの流れに位置付けられる.
分散表現の代表例である Word2Vec\cite{church2017word2vec}は,単語の共起情報に基づいて単語をベクトル空間に埋め込み,意味的に類似した単語どうしが近接するような表現を学習する手法である.
このような単語ベクトルは,類似語検索やクラスタリングのみならず,文書分類や機械翻訳といった下流タスクの入力特徴としても広く利用されており,大規模言語モデルにおける埋め込み層の設計にもつながっている.

この分散表現の枠組みは,テキスト以外のデータにも応用されている.
Shoji ら\cite{shoji2024area}が提案した Area2Vec は,都市内のエリアを滞在履歴に基づいてベクトル空間に埋め込み,互いに類似した利用パターンを持つエリアどうしを近接させる手法である.
さらに,Shoji ら\cite{shoji2025life}は,Area2Vec をはじめとするエリア埋め込みを用いて生活行動パターンを抽出し,都市内での活動特性の違いを分析している.
このように,対象(単語,エリア,行動など)を潜在空間に埋め込むことで,構造や関係性を可視化し,下流の分析や予測に活用するアプローチが一般化しつつある.

スポーツ分野においても,埋め込みモデルを用いた試みが報告されている.
Zhang ら\cite{zhang2019player2vec}は,試合中のプレーシーケンスに基づいて選手を潜在ベクトルとして表現する Player2Vec を提案し,プレースタイルの類似性や戦術上の役割を分析可能にしている.
また,購買データの分野では,Kohei ら\cite{kohei2022}がトランザクションデータを対象に,商品や顧客をベクトル空間に埋め込むことで,類似商品・類似顧客の抽出や需要予測などのタスクに応用できることを示している.

これらの先行研究は,高次元で複雑な構造を持つ対象を低次元のベクトルとして埋め込み,類似性の可視化や下流タスクの性能向上に結び付ける枠組みの有効性を示している.
一方で,チケット販売において,価格帯やファン層,優待・招待といった販売構成そのものを試合単位の潜在ベクトルとして表現し,販売傾向の類似性の分析と来場予測の両方に活用する手法は,十分には検討されていない.
本研究で提案する Game2Vec は,こうした分散表現の考え方をチケット販売データに適用し,販売構成に基づく試合ベクトルを構築し,販売傾向の比較と下流の予測タスク双方の支援を目指す.


\section{本研究の位置付け}

従来研究は「外的要因に基づく最終来場数の予測」,「販売過程の時間推移のモデリング」,「価格帯やファン層といった販売構成の分析」といった個別の側面については多くの知見を蓄積しているが,価格帯,来場頻度に基づくファン層といった販売構成を一体的に扱いつつ,
それが販売期間の中でどのように形成・変化していくのかを試合単位でモデリングする手法は,まだ提案されていない.
そこで本研究では,このギャップに対して,価格帯とファン層からなる 25 カテゴリ別販売率に基づいて試合ごとの販売構成を潜在ベクトルとして表現し,
初動販売と最終販売の関係性の分析および来場予測への活用を可能にする Game2Vec を提案する.

