%%%%%%%%%%%%%%%%%%%%%%%%%%%%%%%%  第4章  %%%%%%%%%%%%%%%%%%%%%%%%%%%%%%%%%%%%%%%%
\chapter{tips}\label{ch:4}
ぜひここを充実させてもらいたい!

\section{文章関連のtips}
\subsection{本文と目次で表示される文章を変えたい}
すでに出ていますが,\url{\section[目次用の文]{本文用の文}}とすることで設定可能です.図や表のキャプションでも同じことができます.

\subsection{refで混乱しないために}
\url{ref{}}で何を参照したのか混乱しないためには,labelの時点で例えば図はfig:をつける,表はtb:とするなどのルールを決めておくと良いでしょう.

\subsection{色を変更する}
プリアンプルでcolorが読み込まれた状態で\textcolor{red}{赤くなあれ}でOKです.

\subsection{範囲コメントアウトを行う}
一行だけのコメントアウトは\url{%}で可能ですが,複数行を一気にコメントアウトする場合は,2つの方法が考えられます.まずひとつ目は,\url{\if0}と\url{\fi}で囲んでしまう方法です.この方法は囲った範囲内に\url{\if}があると失敗する可能性があります.\par
↓ソースコードではここにコメントアウトされたエリアがあります.コンパイル済みのpdfではおそらくこの文章の次に「もうひとつの〜」と文が来るはずです.\par

\if0
ここは
出力
されません.
\fi

もうひとつの方法は,commentパッケージを使う方法です.プリアンプルに\\\url{\usepackage{comment}}を記載しておきます.あとはコメントアウトしたい範囲を\url{\begin{comment}}と\url{\end{comment}}で囲いましょう.

\begin{comment}
ここは
出力
されません.
\end{comment}


\section{図関連のtips}
\subsection{図とキャプションの間を詰めたい・広げたい}
\url{vspace}の出番です.図\ref{fig:change_vspace}に例を示します.図\ref{fig:narrow_vspace}では標準より狭いvspaceを設定しました.図\ref{fig:normal_vspace}と見比べてください.

\begin{figure}[tbh]
  \centering %これ必要かわからない
  % 最初の図---------------------------
  \begin{minipage}[b]{0.45\hsize}
    \centering
    \includegraphics[width=5cm]{images/dummy.png}
    \vspace{-0.2cm}
    \subcaption{短いvspace}
    \label{fig:narrow_vspace}
  \end{minipage}
  % 2番目の図--------------------------
  \begin{minipage}[b]{0.45\hsize}
    \centering
    \includegraphics[ width=5cm]{images/dummy.png}
    \subcaption{標準のvspace}
    \label{fig:normal_vspace}
  \end{minipage}
  \caption{vspaceを調整する}
  \label{fig:change_vspace}
\end{figure}

\subsection{図を回転させたい}
単体の図を回転させるならincludegraphicsのオプションに角度を指定します.結果は図\ref{fig:rotate}のようになります.めったに使うことはないと思いますが.\par
一方こちらは覚えておくと便利かもしれません.ページをまるまる回転させるならlandscapeです.プリアンプルに\url{\usepackage{landscape}}を入れておきます.あとは回転させたい部分を\url{\begin{landspace}}と\url{\end{landscape}}で囲むだけです.表でもなんでも回転できると思います.図\ref{fig:rotate_page}はページごと回転しています.

\begin{figure}[tbh]
  \centering
  \includegraphics[width=5cm, angle=60]{images/dummy.png}
  \caption{60度回転した図}
  \label{fig:rotate}
\end{figure}

\begin{landscape}
  \begin{figure}[tbh]
    \centering
    \includegraphics[width=10cm]{images/dummy.png}
    \caption{ページごと回転した図}
    \label{fig:rotate_page}
  \end{figure}
\end{landscape}

\section{表関連のtips}
\url{booktabs.sty}が利用可能な場合,\url{\toprule}や\url{\bottomrule}を使ってより見た目の良い表を作成できるでしょう.